\section*{\Large\centering \textbf{ CHAPTER 2 \\ BACKGROUND STUDY AND LITERATURE REVIEW }}
\addcontentsline{toc}{section}{CHAPTER 2: Background Study and Literature Review}
\setcounter{section}{2}
\setcounter{subsection}{0}

 \subsection{Background Study}
 Brain tumors remain one of the most critical health challenges globally due to their high mortality and morbidity rates. The global incidence of brain and other central nervous system (\gls{cns}) tumors is approximately 6.2 cases per 100,000 individuals annually, including both malignant and benign types~\cite{tisch2024}. In 2019, there were 347,992 new cases of brain cancer, with males comprising 54\% and females 46\% of the reported cases~\cite{ilic2023}. These tumors account for about 1.9\% of all cancers worldwide, ranking as the 19th most common malignancy and the 12th leading cause of cancer-related deaths, contributing to 2.5\% of total cancer fatalities~\cite{ilic2023}. Such statistics highlight the urgent need for reliable diagnostic solutions to improve early detection and treatment outcomes.

Traditional methods of brain tumor diagnosis rely heavily on manual interpretation of Magnetic Resonance Imaging (\gls{mri}) scans by radiologists. While \gls{mri} provides high-quality, detailed images of brain structures, the manual process is often time-consuming, inconsistent, and prone to human error. These limitations are especially critical in the early stages of tumor development when timely and accurate diagnosis is essential. With the increasing volume of medical imaging data, the demand for automated and efficient diagnostic tools has become more pressing.

Recent advancements in artificial intelligence, particularly deep learning, have introduced powerful solutions for medical imaging. Convolutional Neural Networks (\gls{cnn}s) have shown remarkable success in identifying complex patterns within images, making them suitable for brain tumor detection and classification. By leveraging transfer learning from pre-trained models like \gls{vgg16}, \gls{cnn}s can achieve high accuracy with limited medical datasets, reducing training costs and improving generalization. Moreover, automated tumor segmentation techniques powered by \gls{cnn}s allow precise localization of tumor boundaries, supporting treatment planning and surgical interventions. Together, \gls{cnn}s and transfer learning form the foundation of modern \gls{ai}-based diagnostic systems, offering significant potential to enhance the accuracy, efficiency, and reliability of brain tumor detection.

\subsection{Literature Review}
Convolutional Neural Networks (\gls{cnn}s) have significantly advanced the field of medical image analysis, particularly in brain tumor classification from \gls{mri} scans. These models automatically extract spatial hierarchies of features from pixel data, eliminating the need for manual feature engineering. Transfer learning, especially using architectures like \gls{vgg16} pretrained on ImageNet, has been widely adopted due to limited labeled medical datasets. While some studies report classification accuracies exceeding 97\% using fine-tuned \gls{vgg16} models\cite{mathivanan2024}\cite{babu2023}\cite{khaliki2024}, several others have encountered challenges in achieving such high accuracy due to dataset limitations, overfitting, and task complexity.

For instance, Zohra et al. (2024) compared a custom \gls{cnn} and a \gls{vgg16}-based model on a small, imbalanced brain \gls{mri} dataset and reported test accuracies of only 72\% and 75\%, respectively. The authors attributed this to the small dataset size and class imbalance, which led to overfitting and poor generalization, even though the training accuracy exceeded 98\%\cite{zohra2024}. Similarly, Srinivasan et al. (2024) proposed a hybrid deep \gls{cnn} for five-class brain tumor classification, achieving an overall test accuracy of 93.81\%, with the lowest per-class accuracy of 95.6\% for pituitary tumors. The increased difficulty of multi-class classification and the limited number of training samples for some tumor types were cited as contributing factors\cite{srinivasan2024}.

Aksoy (2025) also employed a \gls{vgg16}-based transfer learning approach on a binary classification task using a private \gls{mri} dataset of 7,000 images. While the model achieved perfect training accuracy, the validation accuracy plateaued at 94\%, and the test accuracy remained slightly below 95\%, highlighting issues with overfitting and potential limitations of transfer learning in handling medical images\cite{aksoy2025}. These findings demonstrate that while \gls{cnn}s and \gls{vgg16} offer powerful tools for tumor analysis, achieving state-of-the-art performance consistently requires robust datasets, proper augmentation, and careful tuning of model architectures.